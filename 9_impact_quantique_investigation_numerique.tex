\chapter{Impact du Quantique sur l'Investigation Numérique}
\epigraph{"Quantum computing will change everything we know about digital security and forensics. Preparation isn't optional—it's essential."}{- Whitfield Diffie}
\section{La Menace Quantique}
\subsection{Algorithme de Shor et ses Implications}
L'algorithme de Shor (1994) peut factoriser de grands nombres en temps polynomial sur un ordinateur quantique, menaçant:

\begin{itemize}
\item \textbf{RSA}: Cassé avec ~4000 qubits logiques
\item \textbf{ECC}: Cassé avec ~2000 qubits logiques
\item \textbf{DSA/ECDSA}: Vulnérables de manière similaire
\end{itemize}

\textbf{Timeline de la Menace} (selon le NIST):

\begin{itemize}
\item 2030: Ordinateurs quantiques de 100-1000 qubits physiques
\item 2035: Menace crédible contre RSA-2048
\item 2040: Cryptographie actuelle obsolète
\end{itemize}

\subsection{Algorithme de Grover et la Recherche}
Accélération quadratique pour:

\begin{itemize}
\item Recherche dans les bases de données
\item Cassage de clés symétriques (AES-128 → sécurité 64-bit)
\item Rainbow tables quantiques
\end{itemize}

\section{Implications pour l'Investigation}
\subsection{"Harvest Now, Decrypt Later"}
\textbf{Problématique actuelle}:

\begin{itemize}
\item Les adversaires stockent des communications chiffrées
\item Attente de l'avènement quantique pour décryptage
\item Impact sur les preuves numériques historiques
\end{itemize}

\textbf{Contre-mesures}:

\begin{lstlisting}[language=Python, caption=Migration vers la crypto hybride]
def hybrid_encryption(data, recipient_public_key):
    """
    Chiffrement hybride classique + post-quantique
    """
    # Classical layer (for current security)
    rsa_encrypted = rsa_encrypt(data, recipient_public_key)
    
    # Post-quantum layer (for future security)
    kyber_encrypted = kyber_encrypt(rsa_encrypted,
                                  recipient_kyber_key)
    
    # Double encryption provides defense in depth
    return kyber_encrypted
\end{lstlisting}

\subsection{Impact sur la Chain of Custody}
\textbf{Défis}:

\begin{enumerate}
\item \textbf{Signatures numériques}: Migration nécessaire vers PQC
\item \textbf{Timestamps}: Besoin de re-timestamping périodique
\item \textbf{Intégrité long-terme}: Hash functions résistantes
\end{enumerate}

\section{Cryptographie Post-Quantique (PQC)}
\subsection{Standards NIST Round 4}
\textbf{Algorithmes sélectionnés} (Juillet 2022):

\textbf{Signatures}:

\begin{itemize}
\item \textbf{CRYSTALS-Dilithium}: Basé sur les réseaux
\item \textbf{FALCON}: Compact, basé sur NTRU
\item \textbf{SPHINCS+}: Hash-based, stateless
\end{itemize}

\textbf{Key Encapsulation}:

\begin{itemize}
\item \textbf{CRYSTALS-Kyber}: Principal standard
\item \textbf{BIKE}: Code-based (alternative)
\item \textbf{HQC}: Code-based (alternative)
\end{itemize}

\subsection{Implémentation en Investigation}
\begin{lstlisting}[language=Python, caption=Signature post-quantique pour evidence]
from pqcrypto.sign import dilithium2

def sign_evidence_pqc(evidence_hash, private_key):
    """
    Signature Dilithium pour preuve numérique
    """
    # Generate quantum-resistant signature
    signature = dilithium2.sign(private_key, evidence_hash)
    
    # Create evidence package
    evidence_package = {
        'hash': evidence_hash,
        'signature': signature,
        'algorithm': 'CRYSTALS-Dilithium2',
        'timestamp': time.time(),
        'security_level': 'NIST-2 (equivalent AES-128)'
    }
    
    return evidence_package
\end{lstlisting}

\section{Quantum Forensics: Nouvelles Opportunités}
\subsection{Quantum Random Number Analysis}
\begin{lstlisting}[language=Python, caption=Détection de QRNG vs PRNG]
def detect_quantum_randomness(bit_stream):
    """
    Analyse statistique pour détecter l'origine quantique
    """
    tests = {
        'monobit': monobit_test(bit_stream),
        'runs': runs_test(bit_stream),
        'spectral': dft_test(bit_stream),
        'autocorrelation': autocorrelation_test(bit_stream)
    }
    
    quantum_score = sum([
        1 for test in tests.values()
        if test > 0.99  # Very high randomness
    ])
    
    return "QUANTUM" if quantum_score >= 3 else "CLASSICAL"
\end{lstlisting}

\subsection{Quantum State Tomography for Evidence}
Application future: Reconstruction d'états quantiques pour preuves

\begin{itemize}
\item Vérification de quantum fingerprints
\item Authentication quantique inviolable
\item Quantum seal pour evidence bags
\end{itemize}