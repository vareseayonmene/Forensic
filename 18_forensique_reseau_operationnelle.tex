\chapter{Forensique Réseau Opérationnelle}

\epigraph{« Le réseau ne ment jamais, mais il faut savoir l'écouter. Chaque paquet raconte une histoire, chaque flux révèle une intention. »}{- \hfill \textit\textipa{Mal\textepsilon tY\textopeno n}}

\section{Introduction à la Forensique Réseau Moderne}

La forensique réseau représente l'art de reconstituer les activités numériques à partir des traces laissées dans l'infrastructure de communication. Dans un contexte post-quantique, cette discipline évolue pour intégrer des considérations de confidentialité avancées tout en maintenant la fiabilité et l'opposabilité des preuves selon le Trilemme CRO.

\subsection{Paradigmes de la Forensique Réseau}

\begin{enumerate}
\item \textbf{Forensique passive} : Analyse de captures existantes
\item \textbf{Forensique active} : Collecte en temps réel
\item \textbf{Forensique prédictive} : Anticipation basée sur l'IA
\item \textbf{Forensique quantique} : Préparation aux communications quantiques
\end{enumerate}

\section{Capture et Analyse PCAP}

\subsection{Architecture de Capture Haute Performance}

\begin{lstlisting}[language=Python, caption=Système de capture PCAP avec validation d'intégrité]
import dpkt
import socket
import hashlib
import time
from collections import defaultdict

class AdvancedPCAPAnalyzer:
    """
    Analyseur PCAP avancé avec intégration CRO
    """
    
    def __init__(self, pcap_file):
        self.pcap_file = pcap_file
        self.flows = defaultdict(list)
        self.anomalies = []
        self.quantum_indicators = []
        
    def analyze_pcap_with_cro_validation(self):
        """
        Analyse PCAP avec validation selon le Trilemme CRO
        """
        analysis_results = {
            'flow_analysis': self.perform_flow_analysis(),
            'protocol_analysis': self.perform_protocol_analysis(),
            'behavioral_analysis': self.perform_behavioral_analysis(),
            'quantum_readiness': self.assess_quantum_readiness(),
            'cro_compliance': self.evaluate_cro_compliance()
        }
        
        # Génération de preuves ZK-NR pour les flows critiques
        critical_flows = self.identify_critical_flows(analysis_results['flow_analysis'])
        
        for flow in critical_flows:
            zk_proof = self.generate_flow_integrity_proof(flow)
            flow['zk_proof'] = zk_proof
            
        return analysis_results
    
    def perform_flow_analysis(self):
        """
        Analyse détaillée des flux réseau
        """
        with open(self.pcap_file, 'rb') as f:
            pcap = dpkt.pcap.Reader(f)
            
            for timestamp, buf in pcap:
                try:
                    eth = dpkt.ethernet.Ethernet(buf)
                    
                    if isinstance(eth.data, dpkt.ip.IP):
                        ip = eth.data
                        
                        # Identification du flux
                        flow_key = self.create_flow_key(ip)
                        
                        # Analyse du payload
                        payload_analysis = self.analyze_payload(ip.data)
                        
                        # Détection de patterns malveillants
                        malicious_patterns = self.detect_malicious_patterns(ip.data)
                        
                        # Évaluation de l'entropie
                        entropy_score = self.calculate_payload_entropy(ip.data)
                        
                        flow_info = {
                            'timestamp': timestamp,
                            'src_ip': socket.inet_ntoa(ip.src),
                            'dst_ip': socket.inet_ntoa(ip.dst),
                            'protocol': ip.p,
                            'payload_size': len(ip.data),
                            'payload_analysis': payload_analysis,
                            'malicious_patterns': malicious_patterns,
                            'entropy_score': entropy_score,
                            'quantum_indicators': self.scan_quantum_indicators(ip.data)
                        }
                        
                        self.flows[flow_key].append(flow_info)
                        
                except Exception as e:
                    continue
                    
        return self.analyze_flow_patterns()
    
    def detect_covert_channels(self):
        """
        Détection de canaux cachés dans le trafic réseau
        """
        covert_channels = []
        
        # Analyse des timings inter-paquets
        timing_analysis = self.analyze_inter_packet_timings()
        
        # Détection de stéganographie réseau
        for flow_key, packets in self.flows.items():
            # Analyse des champs non-utilisés
            unused_fields = self.analyze_unused_fields(packets)
            
            # Analyse des patterns de taille
            size_patterns = self.analyze_size_patterns(packets)
            
            # Test de randomness sur les payloads
            randomness_test = self.test_payload_randomness(packets)
            
            # Analyse temporelle pour détection de modulation
            temporal_modulation = self.detect_temporal_modulation(packets)
            
            if any([unused_fields['suspicious'], size_patterns['anomalous'], 
                   randomness_test['potential_steganography'], 
                   temporal_modulation['detected']]):
                
                covert_channel = {
                    'flow': flow_key,
                    'detection_methods': {
                        'unused_fields': unused_fields,
                        'size_patterns': size_patterns,
                        'randomness': randomness_test,
                        'temporal_modulation': temporal_modulation
                    },
                    'confidence_level': self.calculate_detection_confidence([
                        unused_fields, size_patterns, randomness_test, temporal_modulation
                    ]),
                    'forensic_impact': self.assess_covert_channel_impact(flow_key)
                }
                
                # Génération de preuve cryptographique de détection
                covert_channel['cryptographic_proof'] = self.create_detection_proof(
                    covert_channel
                )
                
                covert_channels.append(covert_channel)
                
        return covert_channels
    
    def perform_deep_packet_inspection_ai(self):
        """
        DPI avec intelligence artificielle pour détection avancée
        """
        import tensorflow as tf
        
        # Modèle pré-entraîné pour classification de trafic
        model = tf.keras.models.load_model('network_classifier_model.h5')
        
        classified_traffic = []
        
        for flow_key, packets in self.flows.items():
            # Extraction de features pour le ML
            features = self.extract_ml_features(packets)
            
            # Classification du trafic
            classification = model.predict(features.reshape(1, -1))
            
            # Analyse de confiance
            confidence = float(tf.nn.softmax(classification)[0].numpy().max())
            
            # Post-traitement pour validation forensique
            if confidence > 0.8:  # Seuil de confiance élevé
                forensic_validation = {
                    'flow': flow_key,
                    'ai_classification': self.interpret_classification(classification),
                    'confidence': confidence,
                    'features': features.tolist(),
                    'validation_status': 'HIGH_CONFIDENCE',
                    'legal_admissibility': self.assess_ai_evidence_admissibility(
                        classification, confidence
                    )
                }
                
                # Application du protocole ZK-NR pour validation IA
                zk_proof = self.create_ai_validation_proof(forensic_validation)
                forensic_validation['zk_proof'] = zk_proof
                
                classified_traffic.append(forensic_validation)
                
        return classified_traffic
\end{lstlisting}

\subsection{Analyse de Protocoles Chiffrés}

\subsubsection{TLS/SSL Traffic Analysis}

\begin{lstlisting}[language=Python, caption=Analyseur de trafic TLS avec détection post-quantique]
class TLSForensicAnalyzer:
    """
    Analyseur forensique du trafic TLS/SSL
    """
    
    def __init__(self):
        self.tls_flows = []
        self.cipher_suites = {}
        self.certificate_chains = []
        
    def analyze_tls_handshakes(self, pcap_data):
        """
        Analyse des handshakes TLS pour extraction de métadonnées
        """
        tls_analysis = {
            'handshake_analysis': [],
            'cipher_negotiation': [],
            'certificate_validation': [],
            'post_quantum_detection': []
        }
        
        for packet in pcap_data:
            if self.is_tls_packet(packet):
                # Parse du handshake TLS
                tls_info = self.parse_tls_handshake(packet)
                
                # Détection de cipher suites post-quantiques
                pq_detection = self.detect_post_quantum_ciphers(tls_info)
                
                # Analyse de la chaîne de certificats
                cert_analysis = self.analyze_certificate_chain(tls_info['certificates'])
                
                # Évaluation de la sécurité du handshake
                security_assessment = {
                    'protocol_version': tls_info['version'],
                    'cipher_strength': self.assess_cipher_strength(tls_info['cipher_suite']),
                    'perfect_forward_secrecy': self.check_pfs(tls_info['key_exchange']),
                    'quantum_resistance': pq_detection['quantum_resistant'],
                    'certificate_validity': cert_analysis['valid'],
                    'forensic_metadata': self.extract_forensic_metadata(tls_info)
                }
                
                # Application du Trilemme CRO
                cro_evaluation = self.evaluate_tls_with_cro(security_assessment)
                
                tls_analysis['handshake_analysis'].append({
                    'tls_info': tls_info,
                    'security_assessment': security_assessment,
                    'cro_evaluation': cro_evaluation,
                    'pq_detection': pq_detection
                })
                
        return tls_analysis
    
    def detect_tls_anomalies(self, tls_flows):
        """
        Détection d'anomalies dans les communications TLS
        """
        anomalies = []
        
        # Analyse statistique des cipher suites
        cipher_distribution = self.analyze_cipher_distribution(tls_flows)
        
        # Détection de cipher suites obsolètes ou suspects
        for flow in tls_flows:
            anomaly_indicators = {
                'weak_ciphers': self.detect_weak_ciphers(flow['cipher_suite']),
                'certificate_anomalies': self.detect_cert_anomalies(flow['certificates']),
                'timing_anomalies': self.detect_timing_anomalies(flow['handshake_timing']),
                'size_anomalies': self.detect_size_anomalies(flow['packet_sizes']),
                'behavioral_anomalies': self.detect_behavioral_anomalies(flow)
            }
            
            # Score d'anomalie composite
            anomaly_score = self.calculate_composite_anomaly_score(anomaly_indicators)
            
            if anomaly_score > 0.7:  # Seuil d'alerte
                anomaly = {
                    'flow': flow,
                    'anomaly_indicators': anomaly_indicators,
                    'anomaly_score': anomaly_score,
                    'forensic_priority': self.calculate_forensic_priority(anomaly_score),
                    'recommended_actions': self.generate_investigation_recommendations(
                        anomaly_indicators
                    )
                }
                
                # Attestation ZK-NR de l'anomalie
                anomaly['zk_attestation'] = self.create_anomaly_attestation(anomaly)
                
                anomalies.append(anomaly)
                
        return sorted(anomalies, key=lambda x: x['forensic_priority'], reverse=True)
\end{lstlisting}

\section{Log Analysis et SIEM}

\subsection{Analyse Unifiée de Logs}

\begin{lstlisting}[language=Python, caption=Analyseur unifié de logs avec corrélation intelligente]
class UnifiedLogAnalyzer:
    """
    Analyseur unifié intégrant multiples sources de logs
    """
    
    def __init__(self):
        self.log_parsers = {
            'syslog': self.parse_syslog,
            'windows_event': self.parse_windows_events,
            'apache': self.parse_apache_logs,
            'nginx': self.parse_nginx_logs,
            'firewall': self.parse_firewall_logs,
            'ids': self.parse_ids_logs,
            'database': self.parse_database_logs
        }
        self.correlation_engine = CorrelationEngine()
        
    def analyze_multi_source_logs(self, log_sources):
        """
        Analyse corrélée de sources multiples de logs
        """
        parsed_logs = {}
        
        # Parsing de chaque source
        for source_name, source_path in log_sources.items():
            if source_name in self.log_parsers:
                parsed_logs[source_name] = self.log_parsers[source_name](source_path)
                
                # Enrichissement avec métadonnées forensiques
                for log_entry in parsed_logs[source_name]:
                    log_entry['source'] = source_name
                    log_entry['forensic_value'] = self.assess_forensic_value(log_entry)
                    log_entry['cro_metrics'] = self.calculate_log_cro_metrics(log_entry)
                    
        # Corrélation cross-source
        correlations = self.correlation_engine.correlate_across_sources(parsed_logs)
        
        # Construction de la timeline maître
        master_timeline = self.build_master_timeline(parsed_logs)
        
        # Détection de patterns d'attaque
        attack_patterns = self.detect_attack_patterns(correlations, master_timeline)
        
        # Analyse de la chaîne d'attaque (Kill Chain)
        kill_chain_analysis = self.analyze_kill_chain(attack_patterns)
        
        return {
            'parsed_logs': parsed_logs,
            'correlations': correlations,
            'master_timeline': master_timeline,
            'attack_patterns': attack_patterns,
            'kill_chain_analysis': kill_chain_analysis,
            'investigation_recommendations': self.generate_investigation_recommendations(
                attack_patterns
            )
        }
    
    def detect_log_tampering(self, log_file):
        """
        Détection de manipulation de logs
        """
        tampering_indicators = {
            'timestamp_anomalies': self.detect_timestamp_anomalies(log_file),
            'missing_entries': self.detect_missing_log_entries(log_file),
            'hash_validation': self.validate_log_hashes(log_file),
            'sequence_validation': self.validate_log_sequence(log_file),
            'format_anomalies': self.detect_format_anomalies(log_file)
        }
        
        # Score de confiance dans l'intégrité
        integrity_score = self.calculate_log_integrity_score(tampering_indicators)
        
        # Génération de rapport de tampering
        tampering_report = {
            'file': log_file,
            'indicators': tampering_indicators,
            'integrity_score': integrity_score,
            'confidence_level': self.calculate_confidence_level(tampering_indicators),
            'legal_implications': self.assess_legal_implications(integrity_score),
            'remediation_recommendations': self.generate_remediation_recommendations(
                tampering_indicators
            )
        }
        
        # Attestation ZK-NR de l'analyse d'intégrité
        if integrity_score < 0.8:  # Suspicion de tampering
            tampering_report['zk_attestation'] = self.create_tampering_attestation(
                tampering_report
            )
            
        return tampering_report
    
    def analyze_dns_forensics(self, dns_logs):
        """
        Analyse forensique DNS avancée
        """
        dns_analysis = {
            'domain_analysis': self.analyze_domain_patterns(dns_logs),
            'dga_detection': self.detect_domain_generation_algorithms(dns_logs),
            'dns_tunneling': self.detect_dns_tunneling(dns_logs),
            'c2_communication': self.detect_c2_dns_patterns(dns_logs),
            'exfiltration_detection': self.detect_dns_exfiltration(dns_logs)
        }
        
        # Analyse temporelle des requêtes DNS
        temporal_analysis = self.analyze_dns_temporal_patterns(dns_logs)
        
        # Corrélation avec threat intelligence
        ti_correlation = self.correlate_with_threat_intelligence(dns_analysis)
        
        # Évaluation selon CRO
        for domain_info in dns_analysis['domain_analysis']:
            domain_info['cro_assessment'] = self.assess_domain_cro_impact(domain_info)
            
        return {
            'dns_analysis': dns_analysis,
            'temporal_analysis': temporal_analysis,
            'threat_intelligence': ti_correlation,
            'forensic_conclusions': self.generate_dns_forensic_conclusions(dns_analysis)
        }
\end{lstlisting}

\subsection{Détection Avancée d'Intrusions}

\subsubsection{Corrélation Comportementale Multi-Source}

\begin{lstlisting}[language=Python, caption=Moteur de corrélation comportementale]
class BehavioralCorrelationEngine:
    """
    Moteur de corrélation comportementale pour détection d'intrusions
    """
    
    def __init__(self):
        self.behavior_baselines = {}
        self.anomaly_thresholds = {
            'network': 0.15,
            'process': 0.10,
            'file': 0.20,
            'user': 0.25
        }
        
    def establish_behavioral_baselines(self, historical_data):
        """
        Établissement de baselines comportementales
        """
        for data_type, data_samples in historical_data.items():
            # Calcul des métriques statistiques
            baseline_metrics = {
                'mean_activity': np.mean([s['activity_level'] for s in data_samples]),
                'std_deviation': np.std([s['activity_level'] for s in data_samples]),
                'typical_patterns': self.extract_typical_patterns(data_samples),
                'temporal_patterns': self.extract_temporal_patterns(data_samples),
                'user_patterns': self.extract_user_patterns(data_samples)
            }
            
            # Application de techniques d'apprentissage automatique
            ml_baseline = self.create_ml_baseline(data_samples)
            
            self.behavior_baselines[data_type] = {
                'statistical_baseline': baseline_metrics,
                'ml_baseline': ml_baseline,
                'confidence_interval': self.calculate_confidence_interval(data_samples),
                'last_updated': time.time()
            }
            
    def detect_behavioral_anomalies(self, current_data):
        """
        Détection d'anomalies comportementales en temps réel
        """
        anomalies = []
        
        for data_type, current_samples in current_data.items():
            if data_type not in self.behavior_baselines:
                continue
                
            baseline = self.behavior_baselines[data_type]
            
            # Comparaison statistique
            statistical_deviation = self.calculate_statistical_deviation(
                current_samples, baseline['statistical_baseline']
            )
            
            # Prédiction ML
            ml_anomaly_score = baseline['ml_baseline'].decision_function(
                [self.extract_ml_features([current_samples])]
            )[0]
            
            # Score composite d'anomalie
            composite_score = self.calculate_composite_anomaly_score(
                statistical_deviation, ml_anomaly_score
            )
            
            if composite_score > self.anomaly_thresholds[data_type]:
                anomaly = {
                    'data_type': data_type,
                    'anomaly_score': composite_score,
                    'statistical_deviation': statistical_deviation,
                    'ml_score': ml_anomaly_score,
                    'contributing_factors': self.identify_contributing_factors(
                        current_samples, baseline
                    ),
                    'forensic_significance': self.assess_forensic_significance(
                        composite_score, data_type
                    ),
                    'investigation_priority': self.calculate_investigation_priority(
                        composite_score, data_type
                    )
                }
                
                # Génération de preuve cryptographique d'anomalie
                anomaly['cryptographic_proof'] = self.create_anomaly_proof(anomaly)
                
                anomalies.append(anomaly)
                
        return sorted(anomalies, key=lambda x: x['investigation_priority'], reverse=True)
\end{lstlisting}

\section{Threat Hunting sur Réseaux}

\subsection{Hunting Proactif avec Intelligence Artificielle}

\begin{lstlisting}[language=Python, caption=Système de threat hunting proactif]
class ProactiveThreatHunter:
    """
    Système de threat hunting proactif pour environnements réseau
    """
    
    def __init__(self, network_sensors):
        self.sensors = network_sensors
        self.hunting_hypotheses = []
        self.iocs = []
        self.behavioral_models = {}
        
    def generate_hunting_hypotheses(self, threat_intelligence):
        """
        Génération d'hypothèses de hunting basées sur la TI
        """
        hypotheses = []
        
        for threat in threat_intelligence['current_threats']:
            # Analyse des TTPs (Tactics, Techniques, Procedures)
            ttps = threat['mitre_attack_mapping']
            
            # Génération d'hypothèses spécifiques
            for ttp in ttps:
                hypothesis = {
                    'id': f"HYP-{threat['id']}-{ttp['technique_id']}",
                    'description': f"Recherche de {ttp['technique_name']} "
                                 f"associé à {threat['actor_name']}",
                    'detection_logic': self.create_detection_logic(ttp),
                    'data_sources': self.identify_required_data_sources(ttp),
                    'expected_indicators': self.generate_expected_indicators(ttp),
                    'confidence_threshold': self.calculate_confidence_threshold(ttp),
                    'false_positive_mitigation': self.create_fp_mitigation_strategy(ttp)
                }
                
                hypotheses.append(hypothesis)
                
        return hypotheses
    
    def execute_hunting_campaign(self, hypotheses):
        """
        Exécution d'une campagne de threat hunting
        """
        hunting_results = []
        
        for hypothesis in hypotheses:
            # Collecte de données selon l'hypothèse
            relevant_data = self.collect_hypothesis_data(hypothesis)
            
            # Application de la logique de détection
            detection_results = self.apply_detection_logic(
                hypothesis['detection_logic'], relevant_data
            )
            
            # Évaluation des résultats
            for result in detection_results:
                confidence_score = self.calculate_detection_confidence(
                    result, hypothesis['confidence_threshold']
                )
                
                if confidence_score > hypothesis['confidence_threshold']:
                    # Analyse approfondie de la détection
                    deep_analysis = self.perform_deep_analysis(result)
                    
                    # Application du Trilemme CRO
                    cro_analysis = self.apply_cro_to_detection(result, deep_analysis)
                    
                    # Génération de preuve ZK-NR pour la détection
                    zk_proof = self.create_detection_proof(result, deep_analysis)
                    
                    hunting_finding = {
                        'hypothesis': hypothesis['id'],
                        'detection_result': result,
                        'confidence_score': confidence_score,
                        'deep_analysis': deep_analysis,
                        'cro_analysis': cro_analysis,
                        'zk_proof': zk_proof,
                        'next_actions': self.recommend_next_actions(result),
                        'escalation_level': self.determine_escalation_level(confidence_score)
                    }
                    
                    hunting_results.append(hunting_finding)
                    
        return self.prioritize_hunting_results(hunting_results)
    
    def analyze_lateral_movement_patterns(self, network_logs):
        """
        Analyse des patterns de mouvement latéral
        """
        movement_analysis = {
            'credential_reuse': self.detect_credential_reuse(network_logs),
            'authentication_patterns': self.analyze_auth_patterns(network_logs),
            'privilege_escalation': self.detect_privilege_escalation(network_logs),
            'persistence_mechanisms': self.detect_persistence_mechanisms(network_logs),
            'c2_beaconing': self.detect_c2_beaconing(network_logs)
        }
        
        # Construction de graphes de mouvement
        movement_graph = self.build_movement_graph(movement_analysis)
        
        # Identification des chemins d'attaque
        attack_paths = self.identify_attack_paths(movement_graph)
        
        # Évaluation de l'impact
        impact_assessment = self.assess_lateral_movement_impact(attack_paths)
        
        return {
            'movement_analysis': movement_analysis,
            'movement_graph': movement_graph,
            'attack_paths': attack_paths,
            'impact_assessment': impact_assessment,
            'mitigation_recommendations': self.generate_mitigation_recommendations(
                attack_paths
            )
        }
\end{lstlisting}

\section{Attribution Technique d'Attaques}

\subsection{Méthodologie d'Attribution Multi-Dimensionnelle}

L'attribution d'attaques constitue l'un des défis les plus complexes de la forensique réseau, nécessitant une approche multi-dimensionnelle intégrant techniques traditionnelles et innovations post-quantiques.

\begin{lstlisting}[language=Python, caption=Système d'attribution multi-dimensionnel]
class MultiDimensionalAttributionSystem:
    """
    Système d'attribution d'attaques multi-dimensionnel
    """
    
    def __init__(self):
        self.attribution_dimensions = {
            'technical': TechnicalAttributionEngine(),
            'behavioral': BehavioralAttributionEngine(),
            'linguistic': LinguisticAttributionEngine(),
            'temporal': TemporalAttributionEngine(),
            'operational': OperationalAttributionEngine()
        }
        self.threat_actors_db = ThreatActorsDatabase()
        
    def perform_comprehensive_attribution(self, attack_data):
        """
        Attribution complète multi-dimensionnelle
        """
        attribution_results = {}
        
        # Analyse par dimension
        for dimension_name, engine in self.attribution_dimensions.items():
            dimension_analysis = engine.analyze(attack_data)
            
            # Validation de la fiabilité de l'analyse
            reliability_score = self.validate_analysis_reliability(
                dimension_analysis, dimension_name
            )
            
            # Application du Trilemme CRO
            cro_assessment = self.assess_dimension_cro_impact(
                dimension_analysis, dimension_name
            )
            
            attribution_results[dimension_name] = {
                'analysis': dimension_analysis,
                'reliability_score': reliability_score,
                'cro_assessment': cro_assessment,
                'weight': self.calculate_dimension_weight(
                    dimension_name, reliability_score
                )
            }
            
        # Fusion des analyses
        fused_attribution = self.fuse_attribution_analyses(attribution_results)
        
        # Comparaison avec base de connaissances
        similarity_scores = self.compare_with_known_actors(fused_attribution)
        
        # Génération de rapport d'attribution
        attribution_report = self.generate_attribution_report(
            fused_attribution, similarity_scores
        )
        
        # Validation cryptographique avec ZK-NR
        attribution_report['cryptographic_validation'] = self.create_attribution_proof(
            attribution_report
        )
        
        return attribution_report
    
    def analyze_infrastructure_attribution(self, network_indicators):
        """
        Attribution basée sur l'analyse d'infrastructure
        """
        infrastructure_analysis = {
            'ip_analysis': self.analyze_ip_infrastructure(network_indicators['ips']),
            'domain_analysis': self.analyze_domain_infrastructure(network_indicators['domains']),
            'certificate_analysis': self.analyze_certificate_infrastructure(
                network_indicators['certificates']
            ),
            'hosting_analysis': self.analyze_hosting_patterns(network_indicators),
            'registration_analysis': self.analyze_registration_patterns(network_indicators)
        }
        
        # Analyse des patterns de réutilisation d'infrastructure
        reuse_patterns = self.analyze_infrastructure_reuse(infrastructure_analysis)
        
        # Corrélation avec attaques connues
        known_attacks_correlation = self.correlate_with_known_infrastructure(
            infrastructure_analysis
        )
        
        # Scoring de confiance
        confidence_scores = {}
        for aspect, analysis in infrastructure_analysis.items():
            confidence_scores[aspect] = self.calculate_infrastructure_confidence(
                analysis, known_attacks_correlation
            )
            
        return {
            'infrastructure_analysis': infrastructure_analysis,
            'reuse_patterns': reuse_patterns,
            'correlations': known_attacks_correlation,
            'confidence_scores': confidence_scores,
            'attribution_candidates': self.identify_attribution_candidates(
                infrastructure_analysis, confidence_scores
            )
        }
\end{lstlisting}

\subsection{Analyse Géospatiale et Temporelle}

\begin{lstlisting}[language=Python, caption=Analyseur géospatial pour attribution]
class GeospatialTemporalAnalyzer:
    """
    Analyseur géospatial et temporel pour attribution d'attaques
    """
    
    def __init__(self):
        self.geolocation_db = GeolocationDatabase()
        self.timezone_analyzer = TimezoneAnalyzer()
        
    def analyze_geographic_patterns(self, network_activity):
        """
        Analyse des patterns géographiques d'activité
        """
        geographic_analysis = {}
        
        # Géolocalisation des adresses IP
        geolocated_ips = []
        for ip in network_activity['source_ips']:
            geolocation = self.geolocation_db.lookup(ip)
            
            if geolocation:
                geolocated_ips.append({
                    'ip': ip,
                    'country': geolocation['country'],
                    'region': geolocation['region'],
                    'city': geolocation['city'],
                    'coordinates': geolocation['coordinates'],
                    'accuracy': geolocation['accuracy'],
                    'activity_times': self.extract_activity_times(ip, network_activity)
                })
                
        # Analyse des clusters géographiques
        geographic_clusters = self.identify_geographic_clusters(geolocated_ips)
        
        # Analyse de la distribution temporelle par région
        temporal_distribution = self.analyze_temporal_distribution_by_region(
            geolocated_ips
        )
        
        # Détection de patterns d'infrastructure partagée
        shared_infrastructure = self.detect_shared_infrastructure_patterns(
            geolocated_ips
        )
        
        # Corrélation avec fuseaux horaires
        timezone_correlation = self.correlate_with_working_hours(temporal_distribution)
        
        return {
            'geolocated_activity': geolocated_ips,
            'geographic_clusters': geographic_clusters,
            'temporal_distribution': temporal_distribution,
            'shared_infrastructure': shared_infrastructure,
            'timezone_correlation': timezone_correlation,
            'attribution_confidence': self.calculate_geographic_attribution_confidence(
                geographic_clusters, timezone_correlation
            )
        }
    
    def perform_traffic_flow_analysis(self, netflow_data):
        """
        Analyse des flux de trafic pour détection d'activités suspectes
        """
        flow_analysis = {
            'volume_analysis': self.analyze_traffic_volumes(netflow_data),
            'pattern_analysis': self.analyze_flow_patterns(netflow_data),
            'anomaly_detection': self.detect_flow_anomalies(netflow_data),
            'beaconing_detection': self.detect_beaconing_patterns(netflow_data),
            'exfiltration_detection': self.detect_data_exfiltration(netflow_data)
        }
        
        # Clustering des flows par similarité
        flow_clusters = self.cluster_similar_flows(netflow_data)
        
        # Analyse des patterns temporels
        temporal_patterns = self.analyze_flow_temporal_patterns(netflow_data)
        
        # Machine Learning pour classification de flows
        ml_classification = self.classify_flows_with_ml(netflow_data)
        
        # Évaluation forensique des résultats
        forensic_evaluation = self.evaluate_flows_forensically(
            flow_analysis, flow_clusters, ml_classification
        )
        
        return {
            'flow_analysis': flow_analysis,
            'flow_clusters': flow_clusters,
            'temporal_patterns': temporal_patterns,
            'ml_classification': ml_classification,
            'forensic_evaluation': forensic_evaluation,
            'investigation_leads': self.generate_investigation_leads(forensic_evaluation)
        }
\end{lstlisting}

\section{Forensique de Protocoles Émergents}

\subsection{Analyse des Communications 5G/6G}

\begin{table}[h]
\centering
\begin{tabular}{|l|l|l|l|}
\hline
\textbf{Protocole 5G} & \textbf{Défi Forensique} & \textbf{Solution CRO} & \textbf{Maturité} \\
\hline
Network Slicing & Isolation forensique & Q2CSI layering & Émergente \\
Edge Computing & Distributed evidence & ZK-NR aggregation & En développement \\
Massive IoT & Volume et hétérogénéité & AI-driven triage & Recherche \\
Ultra-Low Latency & Captures haute fréquence & Streaming analysis & Prototype \\
\hline
\end{tabular}
\caption{Défis forensiques des protocoles 5G/6G}
\end{table}

\subsection{Forensique des Protocoles Post-Quantiques}

\begin{lstlisting}[language=Python, caption=Analyseur de protocoles post-quantiques]
class PostQuantumProtocolAnalyzer:
    """
    Analyseur spécialisé pour protocoles post-quantiques
    """
    
    def __init__(self):
        self.pqc_protocols = {
            'quantum_key_distribution': QKDAnalyzer(),
            'post_quantum_tls': PQTLSAnalyzer(),
            'quantum_secured_vpn': QSVPNAnalyzer(),
            'quantum_safe_messaging': QSMAnalyzer()
        }
        
    def analyze_quantum_communication_patterns(self, network_capture):
        """
        Analyse des patterns de communication quantique
        """
        quantum_patterns = {
            'qkd_sessions': self.detect_qkd_sessions(network_capture),
            'quantum_entanglement_markers': self.detect_entanglement_markers(network_capture),
            'post_quantum_handshakes': self.detect_pq_handshakes(network_capture),
            'quantum_error_correction': self.detect_qec_patterns(network_capture)
        }
        
        # Évaluation de la sécurité quantique
        quantum_security_assessment = self.assess_quantum_security(quantum_patterns)
        
        # Impact sur l'investigation forensique
        forensic_implications = self.assess_quantum_forensic_implications(
            quantum_patterns, quantum_security_assessment
        )
        
        return {
            'quantum_patterns': quantum_patterns,
            'security_assessment': quantum_security_assessment,
            'forensic_implications': forensic_implications,
            'investigation_adaptations': self.recommend_investigation_adaptations(
                forensic_implications
            )
        }
\end{lstlisting}

\section{Conclusion et Perspectives d'Évolution}

La forensique réseau opérationnelle évolue rapidement vers une discipline hautement spécialisée nécessitant :

\begin{itemize}
\item \textbf{Expertise multi-protocole} : Maîtrise des protocoles classiques et émergents
\item \textbf{Intelligence artificielle} : Automatisation de la détection et de l'analyse
\item \textbf{Cryptographie avancée} : Intégration des protocoles post-quantiques
\item \textbf{Validation juridique} : Application systématique du framework ZK-NR
\end{itemize}

L'investigateur réseau moderne doit développer une vision systémique intégrant les aspects techniques, légaux et éthiques de son travail, tout en anticipant les évolutions technologiques futures.

\subsection{Défis Futurs}

\begin{enumerate}
\item \textbf{Quantum Internet} : Préparation aux communications quantiques
\item \textbf{AI-Generated Traffic} : Détection de trafic généré par IA
\item \textbf{Homomorphic Communications} : Analyse sur données chiffrées
\item \textbf{Blockchain Networks} : Forensique des réseaux décentralisés
\end{enumerate}

La maîtrise de ces domaines émergents déterminera l'efficacité des investigations réseau de demain.