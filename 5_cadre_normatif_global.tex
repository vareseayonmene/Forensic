\chapter{Cadre Normatif Global}
\epigraph{"La loi doit fournir un cadre sans entraver l'innovation, protéger sans étouffer, réguler sans paralyser."}{- Simone Veil}
\section{ISO/IEC 27037:2012}
\textbf{"Information technology --- Security techniques --- Guidelines for identification, collection, acquisition and preservation of digital evidence"}

\subsection{Principes Fondamentaux:}
\begin{enumerate}
\item \textbf{Pertinence}: Collecte ciblée et justifiée
\item \textbf{Fiabilité}: Méthodes reproductibles et vérifiables
\item \textbf{Suffisance}: Collecte exhaustive dans le périmètre défini
\item \textbf{Documentation}: Traçabilité complète
\end{enumerate}

\subsection{Application Pratique:}
\begin{verbatim}
Processus de Saisie selon ISO 27037:

1. Identification préliminaire
   - Type de dispositif
   - État (allumé/éteint)
   - Connexions actives

2. Documentation photographique

3. Isolation (mode avion, Faraday)

4. Acquisition (write-blocker obligatoire)

5. Vérification (hash SHA-256 minimum)

6. Scellement et transport
\end{verbatim}

\section{ISO/IEC 27041:2015}
\textbf{"Guidance on assuring suitability and adequacy of incident investigative method"}

\subsection{Méthodes Validées:}
\begin{itemize}
\item \textbf{Live Forensics}: RFC 3227 compliant
\item \textbf{Dead Forensics}: NIST SP 800-86 compliant
\item \textbf{Network Forensics}: IETF standards
\item \textbf{Mobile Forensics}: NIST SP 800-101
\end{itemize}

\section{ISO/IEC 27042:2015}
\textbf{"Guidelines for the analysis and interpretation of digital evidence"}

\subsection{Framework d'Analyse:}
\begin{enumerate}
\item \textbf{Préparation}: Environnement d'analyse isolé
\item \textbf{Extraction}: Récupération de données
\item \textbf{Analyse}: Application de techniques
\item \textbf{Interprétation}: Contextualisation
\item \textbf{Rapport}: Documentation des découvertes
\end{enumerate}

\section{ISO/IEC 27043:2015}
\textbf{"Incident investigation principles and processes"}

\subsection{Modèle de Processus:}
\begin{verbatim}
Readiness → Detection → Initial Response →
Strategy → Collection → Analysis →
Presentation → Post-Investigation
\end{verbatim}

\section{NIST SP 800-86}
\textbf{"Guide to Integrating Forensic Techniques into Incident Response"}

\subsection{Phases Détaillées:}
\begin{enumerate}
\item \textbf{Collection Phase}
\begin{itemize}
\item Data prioritization
\item Evidence preservation
\item Chain of custody
\end{itemize}

\item \textbf{Examination Phase}
\begin{itemize}
\item Data extraction
\item Manual review
\item Automated analysis
\end{itemize}

\item \textbf{Analysis Phase}
\begin{itemize}
\item Timeline reconstruction
\item Correlation
\item Attribution
\end{itemize}

\item \textbf{Reporting Phase}
\begin{itemize}
\item Executive summary
\item Technical details
\item Recommendations
\end{itemize}
\end{enumerate}

\section{RFC 3227 (BCP 55)}
\textbf{"Guidelines for Evidence Collection and Archiving"}

\subsection{Ordre de Volatilité (Farmer \& Venema):}
\begin{enumerate}
\item Registres CPU, cache
\item Mémoire système (RAM)
\item État réseau (tables de routage, ARP)
\item Processus en cours
\item Disque dur
\item Logs système distants
\item Configuration physique
\item Topologie réseau
\end{enumerate}

\section{ACPO Good Practice Guide}
\textbf{"Association of Chief Police Officers - Digital Evidence Guidelines"}

\subsection{Quatre Principes:}
\begin{enumerate}
\item \textbf{Principe 1}: Aucune action ne doit modifier les données
\item \textbf{Principe 2}: Compétence requise si modification nécessaire
\item \textbf{Principe 3}: Audit trail complet
\item \textbf{Principe 4}: Responsabilité de conformité
\end{enumerate}

\section{Standards Émergents}
\subsection{Cloud Forensics}
\begin{itemize}
\item \textbf{ISO/IEC 27050}: Electronic discovery
\item \textbf{CSA Guidelines}: Cloud Security Alliance
\item \textbf{NIST SP 800-201}: Cloud forensics challenges
\end{itemize}

\subsection{IoT Forensics}
\begin{itemize}
\item \textbf{IEEE P2933}: Trusted IoT Data
\item \textbf{ETSI TR 103 939}: IoT testing methodology
\item \textbf{ISO/IEC 30141}: IoT reference architecture
\end{itemize}