\chapter{Fondements Théoriques de l'Investigation Numérique}
\epigraph{"La théorie sans la pratique est vaine, la pratique sans la théorie est aveugle. L'investigation numérique exige les deux."}{- Adaptation d'Emmanuel Kant, \textit{Critique de la raison pure}}
\section{Le Principe de Locard Numérique}
Édmond Locard (1877-1966) a établi que "toute action laisse une trace". En investigation numérique, ce principe se décline en:

\subsection{Traces Primaires}
\begin{itemize}
\item \textbf{Logs système}: Enregistrements horodatés des événements
\item \textbf{Artefacts de registre}: Modifications dans les bases de registre
\item \textbf{Fichiers temporaires}: Cache, swap, hibernation
\end{itemize}

\subsection{Traces Secondaires}
\begin{itemize}
\item \textbf{Métadonnées}: EXIF, timestamps, propriétés de fichiers
\item \textbf{Corrélations réseau}: Flux NetFlow, captures PCAP
\item \textbf{Empreintes comportementales}: Patterns d'utilisation
\end{itemize}

\section{Modèles Théoriques d'Investigation}
\subsection{Le Modèle DFRWS (2001)}
\textbf{Digital Forensic Research Workshop Framework}

\begin{enumerate}
\item \textbf{Identification}: Reconnaissance des incidents
\item \textbf{Préservation}: Isolation et protection des preuves
\item \textbf{Collection}: Acquisition méthodique
\item \textbf{Examination}: Analyse détaillée
\item \textbf{Analysis}: Corrélation et reconstruction
\item \textbf{Presentation}: Rapport et témoignage
\end{enumerate}

\subsection{Le Modèle de Casey (2004)}
\textbf{Enhanced Integrated Digital Investigation Process}

\begin{itemize}
\item Phase 1: Readiness (Préparation)
\item Phase 2: Deployment (Déploiement)
\item Phase 3: Physical Crime Scene (Scène physique)
\item Phase 4: Digital Crime Scene (Scène numérique)
\item Phase 5: Review (Révision)
\end{itemize}

\subsection{Le Modèle ISO/IEC 27037:2012}
\textbf{Normes internationales pour la collecte de preuves}

\begin{itemize}
\item Identification
\item Collection/Acquisition
\item Préservation
\item Documentation
\end{itemize}

\section{Théorie de l'Information Appliquée}
\subsection{Entropie de Shannon}
Application à l'investigation:

\[ H(X) = -\sum p(x_i) \log_2 p(x_i) \]

\begin{itemize}
\item Détection d'anomalies par analyse entropique
\item Identification de données chiffrées ou compressées
\item Analyse de randomness pour détecter la stéganographie
\end{itemize}

\subsection{Distance de Hamming et Similarité}
Utilisation pour:

\begin{itemize}
\item Détection de plagiat de code
\item Identification de variantes de malware
\item Analyse de similarité de documents
\end{itemize}

\section{Théorie des Graphes en Investigation}
\subsection{Analyse de Réseaux Sociaux}
\begin{itemize}
\item \textbf{Centralité}: Identification des acteurs clés
\item \textbf{Clustering}: Détection de communautés
\item \textbf{Propagation}: Traçage de la diffusion d'information
\end{itemize}

\subsection{Analyse de Flux de Données}
\begin{itemize}
\item Modélisation des transferts de données
\item Identification des chemins de fuite
\item Reconstruction de chronologies
\end{itemize}