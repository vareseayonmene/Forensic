\chapter{État de l'Art et Évolution Scientifique}
\epigraph{"La science progresse en faisant danser les faits aux rhythmes de nouvelles théories."}{- Marcel Proust}
\section{Chronologie des Avancées Scientifiques}
\subsection{1979: Première Saisie de Données Informatiques}
\begin{itemize}
\item \textbf{Lieu}: FBI, États-Unis
\item \textbf{Innovation}: Développement du concept de "bit-stream copy"
\end{itemize}

\subsection{1984: Introduction du Concept de "Computer Forensics"}
\begin{itemize}
\item \textbf{Auteur}: Agent spécial du FBI, Dan Farmer
\item \textbf{Publication}: "Computer Forensics: An Introduction"
\end{itemize}

\subsection{1992: Développement de SafeBack}
\begin{itemize}
\item \textbf{Créateur}: Sydex Inc.
\item \textbf{Innovation}: Premier outil commercial d'imagerie forensique
\end{itemize}

\subsection{1998: Création d'EnCase}
\begin{itemize}
\item \textbf{Société}: Guidance Software
\item \textbf{Impact}: Standardisation de facto dans les forces de l'ordre
\end{itemize}

\subsection{2002: Publication du RFC 3227}
\begin{itemize}
\item \textbf{Titre}: "Guidelines for Evidence Collection and Archiving"
\item \textbf{Auteurs}: D. Brezinski, T. Killalea
\item \textbf{Impact}: Première RFC dédiée à l'investigation numérique
\end{itemize}

\subsection{2003: Lancement du Projet Sleuth Kit}
\begin{itemize}
\item \textbf{Créateur}: Brian Carrier
\item \textbf{Innovation}: Suite open-source d'outils forensiques
\end{itemize}

\subsection{2006: Introduction de la Timeline Analysis}
\begin{itemize}
\item \textbf{Auteur}: Kristinn Guðjónsson (log2timeline)
\item \textbf{Impact}: Révolution dans la corrélation temporelle
\end{itemize}

\subsection{2008: Émergence de la Memory Forensics}
\begin{itemize}
\item \textbf{Outil clé}: Volatility Framework
\item \textbf{Créateurs}: AAron Walters et al.
\item \textbf{Innovation}: Analyse de la mémoire vive volatile
\end{itemize}

\subsection{2012: Cloud Forensics}
\begin{itemize}
\item \textbf{Première conférence dédiée}: IEEE CloudCom
\item \textbf{Défis identifiés}: Multi-juridiction, virtualisation, élasticité
\end{itemize}

\subsection{2015: Machine Learning en Forensique}
\begin{itemize}
\item \textbf{Application}: Classification automatique de malware
\item \textbf{Techniques}: Random Forests, SVM, Deep Learning
\end{itemize}

\subsection{2018: Blockchain Forensics}
\begin{itemize}
\item \textbf{Outils}: Chainalysis, CipherTrace
\item \textbf{Application}: Traçage de cryptomonnaies
\end{itemize}

\subsection{2020: Quantum-Safe Forensics}
\begin{itemize}
\item \textbf{Problématique}: Préparation à l'ère post-quantique
\item \textbf{Innovation}: Développement de signatures résistantes
\end{itemize}

\section{Paradigmes Actuels}
\subsection{Digital Forensics as a Service (DFaaS)}
\begin{itemize}
\item Automatisation des processus d'investigation
\item Scalabilité cloud
\item Intelligence artificielle intégrée
\end{itemize}

\subsection{Proactive Forensics}
\begin{itemize}
\item Préparation anticipée des systèmes
\item Logging amélioré
\item Threat hunting continu
\end{itemize}

\subsection{IoT Forensics}
\begin{itemize}
\item \textbf{Défis}: Hétérogénéité, volume, vélocité
\item \textbf{Solutions}: Edge computing forensics
\item \textbf{Standards émergents}: IEEE 1451
\end{itemize}