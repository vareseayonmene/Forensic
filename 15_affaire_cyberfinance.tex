\chapter{L'Affaire CyberFinance Cameroun 2025}
\epigraph{"Whenever you have excluded the impossible, whatever remains, however improbable, must be the truth."}{- Sir Arthur Conan Doyle}
\section{Présentation du Cas}
\subsection{Contexte}
\textbf{Date}: 15 janvier 2025

\textbf{Victime}: CyberFinance Cameroun S.A.

\begin{itemize}
\item Fintech leader au Cameroun
\item 500,000 clients actifs
\item 50 milliards FCFA de transactions/mois
\end{itemize}

\textbf{Incident}: Attaque ransomware sophistiquée

\begin{itemize}
\item Chiffrement de la base clients
\item Exfiltration de données sensibles
\item Demande de rançon: 10 millions EUR en Bitcoin
\item Menace de divulgation des données
\end{itemize}

\subsection{Infrastructure Compromise}
\begin{verbatim}
Architecture réseau de CyberFinance:

Internet ─── Firewall ─┬─ DMZ ─── Web Servers
                       │ └─ API Gateway
                       │
                       ├─ Internal Network
                       │ ├─ Database Servers
                       │ ├─ Application Servers
                       │ └─ Workstations
                       │
                       └─ Management Network
                         ├─ Backup Systems
                         └─ Admin Consoles
\end{verbatim}

\section{Phase 1: Détection et Réponse Initiale}
\subsection{Chronologie de Détection}
\begin{verbatim}
15/01/2025 02:30 - Premières anomalies réseau détectées
15/01/2025 03:15 - Alertes IDS multiples
15/01/2025 04:00 - Découverte du ransomware
15/01/2025 04:30 - Isolation du réseau
15/01/2025 05:00 - Activation du plan de crise
15/01/2025 06:00 - Notification aux autorités
\end{verbatim}

\subsection{Actions Immédiates}
\begin{lstlisting}[language=Python, caption=Script de réponse d'urgence exécuté]
#!/usr/bin/env python3
import subprocess
import datetime
import json

class IncidentResponse:
    def __init__(self):
        self.incident_id = "INC-2025-0115-001"
        self.start_time = datetime.datetime.now()
        self.actions_log = []
    
    def isolate_network(self):
        """Isolation d'urgence du réseau"""
        commands = [
            "iptables -I INPUT -j DROP",
            "iptables -I OUTPUT -m state --state NEW -j DROP",
            "ip link set eth0 down"  # External interface
        ]
        
        for cmd in commands:
            result = subprocess.run(cmd, shell=True, capture_output=True)
            self.log_action(cmd, result.returncode)
    
    def preserve_volatile_data(self):
        """Capture des données volatiles"""
        volatile_cmds = {
            'processes': 'ps aux',
            'connections': 'netstat -antp',
            'memory_map': 'cat /proc/meminfo',
            'logged_users': 'w',
            'open_files': 'lsof'
        }
        
        for key, cmd in volatile_cmds.items():
            output = subprocess.check_output(cmd, shell=True)
            self.save_evidence(key, output)
    
    def create_memory_dump(self):
        """Dump mémoire pour analyse"""
        dump_cmd = "dd if=/proc/kcore of=/evidence/memory.dump"
        subprocess.run(dump_cmd, shell=True)
        self.hash_evidence("/evidence/memory.dump")
\end{lstlisting}

\section{Phase 2: Investigation Technique}
\subsection{Analyse du Ransomware}
\begin{lstlisting}[language=Python, caption=Analyse du sample de ransomware]
class RansomwareAnalysis:
    def __init__(self, sample_path):
        self.sample = sample_path
        self.iocs = []
    
    def static_analysis(self):
        """Analyse statique du malware"""
        # Extraction des strings
        strings_output = subprocess.check_output(
            f"strings {self.sample}", shell=True
        )
        
        # Recherche d'IoCs
        patterns = {
            'bitcoin_address': r'[13][a-km-zA-HJ-NP-Z1-9]{25,34}',
            'onion_address': r'[a-z2-7]{16,56}\.onion',
            'email': r'[a-zA-Z0-9._%+-]+@[a-zA-Z0-9.-]+\.[a-zA-Z]{2,}',
            'ip_address': r'\d{1,3}\.\d{1,3}\.\d{1,3}\.\d{1,3}'
        }
        
        for pattern_name, regex in patterns.items():
            matches = re.findall(regex, strings_output.decode())
            if matches:
                self.iocs.extend(matches)
        
        return self.iocs
    
    def dynamic_analysis(self):
        """Analyse dynamique en sandbox"""
        # Exécution dans Cuckoo Sandbox
        analysis = {
            'file_operations': self.monitor_file_ops(),
            'network_activity': self.monitor_network(),
            'registry_changes': self.monitor_registry(),
            'process_behavior': self.monitor_processes()
        }
        
        return analysis
\end{lstlisting}

\textbf{Résultats de l'analyse}:

\begin{itemize}
\item \textbf{Famille}: Variant de LockBit 3.0
\item \textbf{Chiffrement}: ChaCha20 + RSA-2048
\item \textbf{Persistance}: Tâche planifiée + modification MBR
\item \textbf{C2}: 3 serveurs Tor identifiés
\item \textbf{Exfiltration}: 850 GB via HTTPS fragmenté
\end{itemize}

\subsection{Analyse Post-Quantique}
Application du framework CRO:

\begin{lstlisting}[language=Python, caption=Évaluation CRO de l'incident]
def evaluate_cro_impact():
    incident_metrics = {
        'confidentiality_breach': 0.95,  # Données exfiltrées
        'reliability_impact': 0.80,      # Systèmes compromis
        'legal_opposability': 0.30       # Preuves altérées
    }
    
    # Application du trilemme CRO
    cro_index = max(incident_metrics.values())
    
    recommendations = {
        'immediate': [
            'Deploy ZK-NR for evidence preservation',
            'Implement Q2CSI architecture',
            'Migrate to PQC signatures'
        ],
        'medium_term': [
            'Full PQC migration',
            'Quantum-safe backup strategy',
            'Legal framework update'
        ]
    }
    
    return cro_index, recommendations
\end{lstlisting}

\section{Phase 3: Collecte de Preuves}
\subsection{Méthodologie ISO 27037}
\begin{lstlisting}[language=Python, caption=Acquisition d'image disque selon ISO 27037]
class EvidenceCollection:
    def __init__(self):
        self.evidence_items = []
        self.chain_of_custody = []
    
    def collect_disk_image(self, disk_path):
        """Acquisition d'image disque selon ISO 27037"""
        evidence_id = f"EVD-{datetime.now().strftime('%Y%m%d%H%M%S')}"
        
        # Write-blocker validation
        wb_status = self.verify_write_blocker()
        
        # Acquisition with validation
        acquisition_cmd = f"""
        dcfldd if={disk_path} \\
        of=/evidence/{evidence_id}.dd \\
        hash=sha256 \\
        hashlog=/evidence/{evidence_id}.hash \\
        status=on \\
        statusinterval=1GB
        """
        
        # Post-acquisition verification
        source_hash = self.calculate_hash(disk_path)
        image_hash = self.calculate_hash(f"/evidence/{evidence_id}.dd")
        
        if source_hash == image_hash:
            self.register_evidence(evidence_id, "DISK_IMAGE", "VALID")
        else:
            raise IntegrityError("Hash mismatch!")
\end{lstlisting}

\subsection{Application ZK-NR pour la Preuve}
\begin{lstlisting}[language=Python, caption=Implémentation du protocole ZK-NR pour les preuves]
class ZKNREvidence:
    def __init__(self):
        self.zknr_protocol = ZK_NR_Protocol()
    
    def create_court_admissible_evidence(self, evidence_data):
        """
        Création de preuves opposables avec ZK-NR
        """
        # Layer 1: Iron (Reliability)
        timestamped_evidence = {
            'data': evidence_data,
            'timestamp': self.get_certified_timestamp(),
            'investigator': self.get_investigator_cert(),
            'hash': sha3_256(evidence_data)
        }
        
        # Layer 2: Gold (Confidentiality)
        zk_proof = self.create_zk_proof(
            statement="Evidence E collected according to ISO 27037",
            witness=timestamped_evidence,
            public_input=timestamped_evidence['hash']
        )
        
        # Layer 3: Clay (Legal Opposability)
        legal_attestation = self.get_legal_attestation(
            zk_proof,
            court_jurisdiction="Cameroon",
            legal_framework="Law 2010/012"
        )
        
        return {
            'evidence_package': legal_attestation,
            'admissibility_score': 0.92,
            'cro_metrics': {
                'C': 0.85,  # Sensitive data protected
                'R': 0.95,  # High integrity
                'O': 0.90   # Court admissible
            }
        }
\end{lstlisting}

\section{Phase 4: Analyse Forensique Approfondie}
\subsection{Timeline Reconstruction}
\begin{lstlisting}[language=Python, caption=Reconstruction de la chronologie avec log2timeline]
class TimelineReconstruction:
    def __init__(self, evidence_sources):
        self.sources = evidence_sources
        self.timeline = []
    
    def build_supertimeline(self):
        """Construction d'une super-timeline"""
        # Parse multiple sources
        for source in self.sources:
            if source['type'] == 'windows_evtx':
                self.parse_windows_logs(source['path'])
            elif source['type'] == 'registry':
                self.parse_registry(source['path'])
            elif source['type'] == 'mft':
                self.parse_mft(source['path'])
            elif source['type'] == 'browser':
                self.parse_browser_history(source['path'])
        
        # Correlate and sort
        self.timeline.sort(key=lambda x: x['timestamp'])
        
        # Identify critical events
        critical_events = self.identify_anomalies()
        
        return self.timeline, critical_events
    
    def identify_anomalies(self):
        """Identification des événements suspects"""
        anomalies = []
        
        # Pattern detection
        patterns = {
            'lateral_movement': self.detect_lateral_movement(),
            'data_staging': self.detect_data_staging(),
            'exfiltration': self.detect_exfiltration(),
            'encryption': self.detect_encryption_activity()
        }
        
        return patterns
\end{lstlisting}

\textbf{Timeline critique identifiée}:

\begin{verbatim}
2025-01-14 18:30:15 - Phishing email received (user: comptable@cyberfinance.cm)
2025-01-14 18:45:22 - Malicious attachment executed
2025-01-14 18:46:01 - PowerShell download cradle activated
2025-01-14 18:47:33 - Mimikatz execution detected
2025-01-14 19:15:44 - Lateral movement to DC01
2025-01-14 20:30:11 - Data compression in C:\Windows\Temp
2025-01-14 22:00:00 - Exfiltration begins (HTTPS, 1GB chunks)
2025-01-15 02:00:00 - Ransomware deployment via GPO
2025-01-15 02:30:00 - Encryption process starts
\end{verbatim}

\subsection{Attribution de l'Attaque}
\begin{lstlisting}[language=Python, caption=Analyse selon MITRE ATT&CK]
class ThreatAttribution:
    def __init__(self):
        self.indicators = []
        self.ttps = []  # Tactics, Techniques, Procedures
    
    def analyze_ttps(self):
        """Analyse selon MITRE ATT&CK"""
        observed_ttps = {
            'initial_access': ['T1566.001'],  # Spearphishing Attachment
            'execution': ['T1059.001'],       # PowerShell
            'persistence': ['T1053.005'],     # Scheduled Task
            'privilege_escalation': ['T1068'], # Exploitation
            'defense_evasion': ['T1562.001'], # Disable Security Tools
            'credential_access': ['T1003'],   # Credential Dumping
            'lateral_movement': ['T1021.001'], # RDP
            'collection': ['T1560'],          # Archive Data
            'exfiltration': ['T1041'],        # Exfiltration Over C2
            'impact': ['T1486']               # Data Encrypted
        }
        
        # Compare with known APT groups
        attribution_scores = self.compare_with_known_groups(observed_ttps)
        
        return attribution_scores
    
    def linguistic_analysis(self):
        """Analyse linguistique des notes de rançon"""
        ransom_note = self.extract_ransom_note()
        
        features = {
            'language': detect_language(ransom_note),
            'style': analyze_writing_style(ransom_note),
            'errors': identify_linguistic_patterns(ransom_note),
            'timezone_hints': extract_temporal_patterns(ransom_note)
        }
        
        return features
\end{lstlisting}

\textbf{Résultats d'attribution}:

\begin{itemize}
\item \textbf{Groupe suspecté}: LockBit affiliate "GoldManager"
\item \textbf{Confidence}: 78\%
\item \textbf{Indicateurs}: Réutilisation d'infrastructure, TTP similaires
\item \textbf{Origine probable}: Europe de l'Est (indices linguistiques)
\end{itemize}

\section{Phase 5: Remédiation et Renforcement}
\subsection{Plan de Remédiation}
\begin{lstlisting}[language=Python, caption=Plan de remédiation]
class RemediationPlan:
    def __init__(self):
        self.phases = []
    
    def immediate_actions(self):
        """Actions immédiates (0-48h)"""
        return [
            "Isolate all infected systems",
            "Reset all credentials (passwords, keys, certificates)",
            "Deploy EDR on all endpoints",
            "Implement network segmentation",
            "Enable MFA everywhere",
            "Patch all critical vulnerabilities"
        ]
    
    def short_term_actions(self):
        """Actions court terme (1 semaine)"""
        return [
            "Complete forensic analysis",
            "Rebuild compromised systems from clean backups",
            "Implement SIEM with custom rules",
            "Deploy deception technology (honeypots)",
            "Conduct threat hunting",
            "Review and update IR procedures"
        ]
    
    def long_term_actions(self):
        """Actions long terme (1-6 mois)"""
        return [
            "Implement Zero Trust Architecture",
            "Deploy Q2CSI framework",
            "Migrate to post-quantum cryptography",
            "Establish 24/7 SOC",
            "Implement continuous security testing",
            "Regular tabletop exercises"
        ]
\end{lstlisting}

\subsection{Implémentation Post-Quantique}
\begin{lstlisting}[language=Python, caption=Migration vers une infrastructure post-quantique]
class PostQuantumMigration:
    def __init__(self):
        self.current_crypto = self.audit_current_crypto()
        self.pqc_algorithms = self.select_pqc_algorithms()
    
    def create_migration_plan(self):
        """Plan de migration PQC"""
        migration_phases = {
            'phase1': {
                'duration': '3 months',
                'actions': [
                    'Deploy hybrid TLS (classical + Kyber)',
                    'Implement Dilithium for new certificates',
                    'Test PQC in lab environment'
                ]
            },
            'phase2': {
                'duration': '6 months',
                'actions': [
                    'Migrate critical systems to PQC',
                    'Implement ZK-NR for audit logs',
                    'Deploy quantum-safe backup encryption'
                ]
            },
            'phase3': {
                'duration': '12 months',
                'actions': [
                    'Complete PQC migration',
                    'Implement Q2CSI architecture',
                    'Establish quantum-safe key management'
                ]
            }
        }
        
        return migration_phases
    
    def implement_zknr_logging(self):
        """Implementation du logging ZK-NR"""
        logging_config = {
            'commitment_interval': 60,  # seconds
            'proof_generation': 'STARK',
            'threshold_signers': 5,
            'minimum_signers': 3,
            'storage_backend': 'distributed_ledger',
            'retention_policy': '7_years',
            'legal_compliance': 'CEMAC_regulations'
        }
        
        return ZKNRLogger(logging_config)
\end{lstlisting}

\section{Phase 6: Aspects Juridiques}
\subsection{Procédure Légale au Cameroun}
\begin{verbatim}
Chronologie juridique:

16/01/2025 - Dépôt de plainte au Parquet
17/01/2025 - Ouverture enquête préliminaire
18/01/2025 - Saisine juge d'instruction
20/01/2025 - Commission rogatoire internationale
22/01/2025 - Expertise judiciaire ordonnée
01/02/2025 - Remise rapport d'expertise
15/02/2025 - Audience préliminaire
01/03/2025 - Procès (si arrestation)
\end{verbatim}

\subsection{Préparation du Dossier Judiciaire}
\begin{lstlisting}[language=Python, caption=Préparation du dossier pour le tribunal]
class LegalDossier:
    def __init__(self):
        self.evidence_items = []
        self.expert_reports = []
        self.witness_statements = []
    
    def prepare_court_package(self):
        """Préparation du dossier pour le tribunal"""
        dossier = {
            'executive_summary': self.create_executive_summary(),
            'technical_evidence': self.compile_technical_evidence(),
            'financial_impact': self.calculate_damages(),
            'expert_testimony': self.prepare_expert_testimony(),
            'international_cooperation': self.mlat_requests(),
            'legal_framework': {
                'national': 'Law 2010/012',
                'regional': 'CEMAC Directive 08/08/CM',
                'international': 'Budapest Convention'
            }
        }
        
        # Apply ZK-NR for legal opposability
        for evidence in dossier['technical_evidence']:
            evidence['zknr_attestation'] = self.create_zknr_attestation(
                evidence['data']
            )
            evidence['cro_score'] = self.calculate_cro_score(evidence)
        
        return dossier
    
    def calculate_damages(self):
        """Calcul des préjudices"""
        damages = {
            'direct_losses': {
                'ransom_demand': 5_225_000_000,  # 10M EUR in XAF
                'system_restoration': 500_000_000,
                'forensic_investigation': 150_000_000,
                'legal_fees': 100_000_000
            },
            'indirect_losses': {
                'business_interruption': 2_000_000_000,
                'reputation_damage': 1_000_000_000,
                'customer_compensation': 500_000_000,
                'regulatory_fines': 250_000_000
            },
            'total': 9_725_000_000  # XAF
        }
        
        return damages
\end{lstlisting}

\section{Leçons Apprises et Recommandations}
\subsection{Analyse Post-Mortem}
\begin{lstlisting}[language=Python, caption=Analyse des causes profondes]
class PostMortemAnalysis:
    def __init__(self, incident_data):
        self.incident = incident_data
        self.lessons = []
    
    def root_cause_analysis(self):
        """Analyse des causes profondes"""
        root_causes = {
            'technical': [
                'Outdated email security gateway',
                'Lack of EDR on endpoints',
                'Insufficient network segmentation',
                'Weak password policy',
                'No MFA on critical systems'
            ],
            'human': [
                'Insufficient security awareness training',
                'Lack of phishing simulation exercises',
                'Delayed incident response',
                'Poor communication during crisis'
            ],
            'process': [
                'Outdated incident response plan',
                'No regular backup testing',
                'Lack of threat intelligence integration',
                'Insufficient logging and monitoring'
            ]
        }
        
        return root_causes
    
    def generate_recommendations(self):
        """Génération de recommandations"""
        recommendations = {
            'critical': {
                'timeline': 'Immediate',
                'items': [
                    'Implement Q2CSI architecture',
                    'Deploy ZK-NR for evidence integrity',
                    'Establish 24/7 SOC',
                    'Implement Zero Trust'
                ]
            },
            'high': {
                'timeline': '3 months',
                'items': [
                    'Complete PQC migration for critical systems',
                    'Deploy advanced threat detection',
                    'Implement privileged access management',
                    'Establish threat intelligence program'
                ]
            },
            'medium': {
                'timeline': '6 months',
                'items': [
                    'Complete security awareness program',
                    'Implement security automation',
                    'Establish bug bounty program',
                    'Deploy deception technology'
                ]
            }
        }
        
        return recommendations
\end{lstlisting}

\subsection{Framework de Résilience Post-Quantique}
\begin{lstlisting}[language=Python, caption=Framework de résilience basé sur les contributions de MINKA et al.]
class QuantumResilientFramework:
    """
    Framework de résilience basé sur les contributions
    de MINKA et al. (CRO Trilemma, ZK-NR, Q2CSI)
    """
    
    def __init__(self):
        self.cro_optimizer = CROOptimizer()
        self.zknr_implementation = ZKNRProtocol()
        self.q2csi_architecture = Q2CSIFramework()
    
    def design_resilient_infrastructure(self):
        """
        Conception d'une infrastructure résiliente
        selon le trilemme CRO
        """
        architecture = {
            'evidence_layer': {
                'technology': 'ZK-NR Protocol',
                'cro_balance': {'C': 0.8, 'R': 0.9, 'O': 0.85},
                'features': [
                    'Non-repudiation with privacy',
                    'Post-quantum secure',
                    'Legally admissible',
                    'UC-secure implementation'
                ]
            },
            'operational_layer': {
                'technology': 'Q2CSI Framework',
                'layers': {
                    'iron': 'Temporal integrity and logging',
                    'gold': 'Confidentiality preservation',
                    'clay': 'Institutional anchoring'
                },
                'benefits': [
                    'Dialectical separation of concerns',
                    'Composable security',
                    'Legal explainability'
                ]
            },
            'cryptographic_layer': {
                'algorithms': {
                    'signatures': 'Dilithium-3',
                    'kem': 'Kyber-768',
                    'hash': 'SHA3-256',
                    'zkp': 'STARK'
                },
                'migration_strategy': 'Hybrid progressive'
            }
        }
        
        return architecture
\end{lstlisting}

\section{Conclusion du Cas}
Ce cas pratique illustre l'application concrète de tous les concepts abordés dans ce manuel:

\begin{enumerate}
\item \textbf{Application du Trilemme CRO}: L'incident démontre l'impossibilité de maximiser simultanément C, R, et O.
\item \textbf{Importance du ZK-NR}: Pour créer des preuves opposables tout en préservant la confidentialité.
\item \textbf{Nécessité du Q2CSI}: Architecture en couches pour gérer la complexité.
\item \textbf{Urgence de la migration PQC}: Protection contre les menaces futures.
\item \textbf{Cadre juridique}: Navigation complexe entre juridictions.
\item \textbf{Investigation moderne}: Combinaison de techniques traditionnelles et innovantes.
\end{enumerate}