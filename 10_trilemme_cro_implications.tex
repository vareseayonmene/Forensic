\chapter{Le Trilemme CRO et ses Implications}
\epigraph{"Security is always a trade-off between confidentiality, integrity, and availability. The perfect balance is the holy grail of our field."}{- Bruce Schneier}
\section{Formalisation du Trilemme CRO}
\textbf{Contribution de MINKA MI NGUIDJOI Thierry Emmanuel (ePrint 2025/1348)}

Le Trilemme CRO établit une incompatibilité formelle entre:

\begin{itemize}
\item \textbf{C}onfidentialité: Protection des données sensibles
\item \textbf{R}eliabilité (Fiabilité): Intégrité et authenticité
\item \textbf{O}pposabilité juridique: Valeur probante légale
\end{itemize}

\subsection{Définition Mathématique}
\[
\Gamma_{CRO}(\Pi) = \max\{C(\Pi), R(\Pi), O(\Pi)\} \geq 0.4 + \text{negl}(\lambda)
\]

où:
\begin{itemize}
\item $C(\Pi)$: Indice de confidentialité
\item $R(\Pi)$: Indice de fiabilité
\item $O(\Pi)$: Indice d'opposabilité
\item $\lambda$: Paramètre de sécurité
\end{itemize}

\subsection{Implications Pratiques}
\begin{enumerate}
\item \textbf{Impossibilité de maximisation simultanée}
\item \textbf{Trade-offs nécessaires selon le contexte}
\item \textbf{Besoin d'architectures en couches}
\end{enumerate}

\section{Analyse des Primitives selon CRO}
\subsection{Signatures Classiques}
\begin{lstlisting}[language=Python, caption=Analyse CRO des primitives cryptographiques]
class CRO_Analysis:
    """Analyse CRO des primitives cryptographiques"""
    
    def analyze_rsa_signature(self):
        return {
            'confidentiality': 0.1,  # Pas de confidentialité
            'reliability': 0.8,      # Bonne jusqu'à l'ère quantique
            'opposability': 0.9,     # Excellente actuellement
            'cro_index': 0.6,
            'quantum_resistant': False
        }
    
    def analyze_ring_signature(self):
        return {
            'confidentiality': 0.9,  # Anonymat fort
            'reliability': 0.7,      # Bonne
            'opposability': 0.2,     # Faible (anonymat)
            'cro_index': 0.6,
            'quantum_resistant': False
        }
\end{lstlisting}

\subsection{Zero-Knowledge Proofs}
\textbf{Analyse selon le trilemme}:

\begin{itemize}
\item \textbf{zk-SNARKs}: C=0.9, R=0.7, O=0.4 (trusted setup problématique)
\item \textbf{zk-STARKs}: C=0.8, R=0.8, O=0.6 (transparent, post-quantum)
\item \textbf{Bulletproofs}: C=0.8, R=0.7, O=0.5 (pas de trusted setup)
\end{itemize}

\section{Architecture Q2CSI}
\textbf{Quantum Composable Contextual Security Infrastructure}

(MINKA et al., ePrint 2025/1380)

\subsection{Séparation Dialectique en Couches}
\begin{verbatim}
┌─────────────────────────────────────┐
│ CLAY LAYER (Opposability)           │
│ Institutional Anchoring             │
├─────────────────────────────────────┤
│ GOLD LAYER (Confidentiality)        │
│ Semantic Entropy Preservation       │
├─────────────────────────────────────┤
│ IRON LAYER (Reliability)            │
│ Temporal/Logging Integrity          │
└─────────────────────────────────────┘
\end{verbatim}

\subsection{Implémentation Modulaire}
\begin{lstlisting}[language=Python, caption=Implementation of Q2CSI architecture]
class Q2CSI_Framework:
    """Implementation of Q2CSI architecture"""
    
    def __init__(self):
        self.iron_layer = IronLayer()    # Reliability
        self.gold_layer = GoldLayer()    # Confidentiality
        self.clay_layer = ClayLayer()    # Opposability
    
    def create_evidence(self, data):
        """Create legally admissible evidence"""
        # Layer 1: Ensure reliability
        reliable_data = self.iron_layer.timestamp_and_log(data)
        
        # Layer 2: Add confidentiality
        confidential_proof = self.gold_layer.create_zk_proof(
            reliable_data
        )
        
        # Layer 3: Legal anchoring
        legal_evidence = self.clay_layer.anchor_institutionally(
            confidential_proof
        )
        
        return legal_evidence
\end{lstlisting}