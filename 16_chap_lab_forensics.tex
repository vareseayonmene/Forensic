\chapter{Pratiques Opérationnelles et Gestion d'un Laboratoire Forensique}\label{chap:lab_forensics}
\epigraph{"Every contact leaves a trace."}{- Edmond Locard}
\section{Guide d'Installation et Configuration}\label{sec:setup}
\subsection{Mise en place d’un laboratoire complet}
Description des éléments matériels et logiciels nécessaires à un environnement forensique reproductible.

\subsection{Configuration des environnements SIFT/Remnux/SANS VM}
Procédure d’installation et d’intégration des distributions spécialisées (SIFT Workstation, REMnux, SANS VM).

\subsection{Intégration des outils open source et commerciaux}
Comparaison, compatibilité et recommandations d’hybridation des solutions libres et propriétaires.

\section{Procédures Opérationnelles Standards (SOP)}\label{sec:sop}
\subsection{Checklists d’intervention}
Modèles de listes de vérification à utiliser lors des différentes phases de l’investigation.

\subsection{Modèles de rapports}
Structures standardisées pour la rédaction des rapports techniques et judiciaires.

\subsection{Scripts d’automatisation}
Exemples de scripts facilitant l’acquisition, l’analyse et la documentation des preuves.

\section{Gestion de Laboratoire Forensique}\label{sec:gestion_lab}
\subsection{Infrastructure technique}
Organisation matérielle et logicielle d’un laboratoire conforme aux standards internationaux.

\subsection{Chaîne de custody physique}
Procédures garantissant l’intégrité et la traçabilité des preuves physiques et numériques.

\subsection{Certification et accréditation}
Normes, standards et organismes de certification pertinents pour les laboratoires forensiques.

\section{Formation Pratique Continue}\label{sec:formation}
\subsection{Veille technologique}
Méthodologie pour suivre l’évolution des menaces, outils et standards.

\subsection{Threat intelligence}
Intégration de renseignements sur les menaces dans les pratiques d’investigation.

\subsection{Red team exercises}
Mise en place d’exercices pratiques de simulation pour renforcer les compétences opérationnelles.

\section*{Résumé}
Ce chapitre fournit un cadre opérationnel complet pour la mise en place, la gestion et l’évolution d’un laboratoire forensique. Il associe l’infrastructure technique, les procédures normalisées et la formation continue afin de garantir la conformité, l’efficacité et l’opposabilité juridique des investigations.
